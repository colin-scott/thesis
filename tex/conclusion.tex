Distributed systems, like most software systems, are becoming increasingly complex over time.
In comparison to other areas of software engineering however, the development
tools that help programmers cope with the complexity of distributed \&
concurrent systems are lagging behind their sequential counterparts.
Inspired by the obvious utility of test case reduction tools for sequential
executions, we sought to develop
execution reduction techniques for distributed executions.

In this thesis we investigated two scenarios where execution reduction could be applied to
distributed systems: partially instrumented code, where executions are non-deterministic, and
completely instrumented code, where developers invest engineering effort
to precisely control all relevant sources of non-determinism. We applied our
techniques to 7 different distributed systems, and our results
leave us optimistic that these techniques can be
successfully applied to a wide range of distributed and concurrent systems.

% TODO(cs): add limitations

% TODO(cs): add future work
% TODO(cs): future work: try building up sequences rather than minimizing
% them, using a model checker. (Small model hypothesis.)

% TODO(cs): add concluding remark
% TODO(cs): would be nice to end the paper on a note: what kinds of situations
% we think these techniques will work well for, vs. what kinds of situations
% we think they won't work well for.

