% Between 1 and 3 paragraphs.

% SIGCOMM
% Software bugs are inevitable in software-defined networking control software,
% and troubleshooting
% is a tedious, time-consuming task. In this paper we discuss how to improve
% control software troubleshooting by presenting a technique
% for automatically identifying
% a minimal sequence of inputs responsible for triggering a given bug, without
% making assumptions about the language or instrumentation of the software under
% test. We apply our technique to five open source SDN control
% platforms---Floodlight, NOX, POX, Pyretic, ONOS---and
% illustrate how the minimal causal sequences our system found aided the
% troubleshooting process.
% \colin{Include key insight?}
%
% % NSDI
% When troubleshooting buggy executions of distributed systems, developers typically
% start by manually separating out events that are responsible for triggering the bug (signal) from those that are extraneous (noise).
% We present \sys, a tool for automatically performing this minimization.
% We apply \sys~to buggy executions of two very different distributed systems,
% Raft and Spark, and find that it produces minimized executions that are
% between 1X and 4.6X the size of optimal executions.

When confronted with a buggy execution of a distributed system---which are
commonplace for distributed
systems software---understanding what went wrong requires significant expertise, time, and luck.
As the first step towards fixing the underlying bug, software developers typically
start debugging by manually separating out events that are responsible for triggering the bug (signal) from those that are extraneous (noise).

% since any combination of the
% thousands of concurrent events in the execution may have caused the system to arrive at the unsafe state.
In this thesis, we investigate whether it is possible to automate this
separation process. Our aim is to reduce time and effort spent on
troubleshooting, and we do so by eliminating
events from buggy executions that are not causally related to the bug, ideally producing a ``minimal
causal sequence'' (MCS) of triggering events.

We show that the general problem of execution
minimization is intractable, but we develop, formalize, and
empirically validate a set of heuristics---for both partially instrumented code, and completely instrumented code---that prove effective at
reducing execution size to within a factor of 4.6X of minimal within a bounded
time budget of 12 hours.
To validate our heuristics, we relay our experiences applying our execution reduction
tools to 7 different open source distributed systems.
