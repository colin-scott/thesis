% TODO(cs): one possible way to work in discussion of program analysis related
% work, and to argue for our definition of MCS:
%  - Say: at the extreme, we could require exact same violation exact same
%  code path. But that is basically what these program analysis papers do, and
%  this is problematic because it doesn't remove enough events, i.e. it's too
%  conservative.
%  - On the other extreme, we could only require same violation. But that
%  would be intractable, and not clear that the alternate paths we find are always
%  meaningful. i.e., edit-distance helps us avoid false-positives.
%  - So we have this notion of edit-distance: a middle between those two
%  extremes.

% System Definition
%         States
%         Events
%                 Messages
%                         Internal
%                         External
%                 Modifications
%         Transitions: event takes from one state to another
%         Enabled events
%                 Delivery of sent internal messages
%                 External messages? (enabled by order?)
%                 Modifications? (enabled by order?)
%         Valid schedules
%                 Each event is enabled
%
% Troubleshooting:
%         Invariants
%         Sequence of events and schedule -> invariant violation
%         Want to find small sets of such triggering event sequences
%         “Shrink" formulation:
%                 E, schedule -> Violation?
%         Definition of minimal [subset, not minus one]
%
% Approximations:
%         heuristic schedulers
%         fairness
%         monotonicity conditions?

We start by introducing a model of distributed systems as
groundwork for defining our goals.
As we discuss further in \S\ref{subsec:generality},
we believe this model is general enough to capture the behavior of many
practical systems.
%In \S\ref{dsec:discussion} we
%discuss how the model described here relates to systems in practice.

\subsection{System Model}

Following~\cite{Fischer:1985:IDC:3149.214121}, we model a distributed system as
a collection of $N$ single-threaded processes communicating through messages. Each process $p$ has unbounded
memory, and behaves deterministically according to a transition function of its current state and the messages it receives.
The overall system $S$ is defined by the transition function and initial
configuration for each process.

Processes communicate by sending messages over a network. A message is a pair
$\left(p, m\right)$, where $p$ is the identity of the destination process, and $m$ is the
message value. The network maintains a buffer of pending messages that have
been sent but not yet delivered. Timers are modeled as messages a process can request to be delivered
to itself at a specified later point in the execution.

%Normally, the network is allowed
%to arbitrarily decide when (if at all) to deliver a message in the buffer. However, the
%network may also become partitioned, in which case it is disallowed from
%delivering messages that cross the partition until the partition is healed.

A {\em configuration} of the system consists of the internal state of each process and
the contents of the network's buffer. Initially the network buffer is empty.

An {\em event} moves the system from one configuration to another. Events can
be one of two kinds. {\em Internal events} take place by
removing a message $m$
from the network's buffer and delivering it to the destination $p$. Then,
depending on $m$ and $p$'s internal state, $p$ enters a new internal state
determined by its transition function, and
sends a finite set of messages to other processes. Since processes are
deterministic, internal transitions are completely determined by the contents
of $m$ and $p$'s
state.

% TODO(cs): Not including partitions as
% events may in fact be a way to motivate bounded edit-distance -- to emulate
% the same partition, we need to stay close to original execution.
% At the very least, we should put that thought in as a footnote.
Events can also be {\em external}.
The three external events we consider are:
process starts, which create a new process; forced restarts (crash-recoveries), which force a process to its initial
state (though it may maintain non-volatile state); and external
message sends $\left(p, m\right)$, which insert a message sent from outside the system
into the network buffer (which may be delivered later as an internal event).
We do not need to explicitly model fail-stop failures, since these are
equivalent to permanently partitioning a process from all other processes.

% \fixme{how to model changes in environment?  These don't necessarily invoke a
% "step". So maybe we should separate the notions of a "step" from "external events"}

A {\em schedule} is a finite sequence $\tau$ of events (both external and
internal) that can be applied, in
turn, starting from an initial configuration. Applying each event in the
schedule results in an {\em execution}. We say that a schedule `contains' a sequence of external
events $E = [ e_1, e_2, \ldots, e_n  ]$ if it includes only those external
events (and no other external events) in the given order.

\subsection{Testing}

An {\em invariant} is a predicate $P$ (a safety
condition) over the internal state of all processes at a particular
configuration $C$. We say that configuration
$C$ violates the invariant if $P\left(C\right)$ is false, denoted $\overline{P}\left(C\right)$.

A {\em test orchestrator} generates sequences of external events
$E = [ e_1, e_2, \ldots, e_n ]$, executes them along with some
(arbitrary) schedule of internal
events, and checks whether any invariants were violated during the execution.
% TODO(cs): Recording performance isn't
% really an issue for most tests.. The flipside of this though is that, if
% the instrumentation is in place during runtime, why not just do model
% checking (and model checking finds minimal results by construction).
The test orchestrator records the external events it injected, the
violation it found, and the
interleavings of internal events that appeared during the execution.
% though recording of internal message deliveries is not strictly necessary.

\subsection{Problem Definition}

We are given a schedule $\tau$ injected by a test orchestrator,\footnote{We
explain how we obtain these schedules in \S\ref{dsec:implementation}.} along
with a specific invariant violation $\overline{P}$ observed at the end of the
test orchestrator's execution.

Our main goal is to find a schedule containing a small sequence of
external (input) events that reproduces the violation $\overline{P}$.
Formally, we define a minimal causal sequence (MCS)
to be a subsequence of external events $E' \sqsubseteq E$ such that there exists
a schedule containing $E'$ that produces
$\overline{P}$, but if we were to remove any
single external event $e$ from $E'$, there would not exist any schedules
shorter\footnote{We limit the number of internal events to ensure
that the search space is finite; any asynchronous
distributed system that requires
delivery acknowledgment is not guaranteed
to stop sending messages~\cite{aguilera1997heartbeat}, essentially because nodes cannot distinguish
between crashes of their peers and indefinite message delays.} than $\tau$
containing $E' - e$ that produce
$\overline{P}$.\footnote{It might be possible to reproduce $\overline{P}$ by
removing multiple events from $E'$,
but checking all combinations is tantamount to enumerating its powerset.
Following~\cite{Zeller:2002:SIF:506201.506206}, we
only seek a 1-minimal subsequence $E'$ instead of a globally minimal
subsequence.}

We start by reducing external (input) events because they are the first level
of abstraction that developers reason about. Occasionally, developers can
understand the root cause simply by examining the external events.

For more difficult bugs, developers typically step through the internal events of
the execution to understand more precisely how the system arrived at the unsafe
state. To help with these cases, we turn to reducing internal events after
the external events have been reduced. At this stage we fix the external events and search for
smaller schedules that still triggers the invariant
violation, for example, by keeping some
messages pending rather than delivering them.
Lastly, we seek to shrink the contents (e.g. data payloads) of external messages.

Note that we do not focus on bugs involving only sequential computation (e.g.
incorrect handling of unexpected input), performance, or human
misconfiguration. Those three bug types
are more common than our focus: concurrency bugs.
We target concurrency bugs because they are the most complex (correspondingly, they take considerably more time to
debug~\cite{msoft_concurrency}), and because mature debugging tools already exist for sequential code.
%systems lies,  % and because there are existing tools for sequential code

With a reduced execution in hand, the developer begins debugging.
Echoing the benefits of sequential test case reduction, we claim that the greatly reduced size of the
trace makes it easier to understand which code path contains
the underlying bug, allowing the developer to focus on
fixing the problematic code itself.

% Our first task is to produce a schedule
% $\tau^{\circ} = e_1\rightarrow i_1\rightarrow \ldots e_2\rightarrow i_m \ldots$
% containing the original external events $E$
% that reproduces the invariant violation $\overline{P}$. If we cannot find a
% schedule that reproduces the violation, then we cannot proceed with
% minimization.

